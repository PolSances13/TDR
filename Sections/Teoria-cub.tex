\chapter{Notació dels Moviments}

El cub de Rubik es resol gràcies a identificar patrons i executar algoritmes que resolen
aquests patrons, aquests algoritmes han d'estar escrits en alguna part per poder-los memoritzar, i per això està la notació del cub de Rubik.
\\\\La notació consta de 6 moviments (F,B,R,L,U,D), que correspon a (Front, Back, Right, Left, Up, Down) que són les respectives direccions en anglès. Per exemple si faig el moviment F, gira la cara front, la que està més propera a la nostra visió, en sentit horari, en canvi, si fós F' seria antihorari. En les figures següents es mostra una representació  gràfica per a cada capa.
És un concepte difícil d'entendre, però de manera simplificada és girar la cara en sentit horari i antihorari des de la cara que vulguis. En les figures següents es mostra una representació gràfica per a cada capa.

\begin{figure}[htbp]
    \centering
    \begin{subfigure}
        \centering\RubikCubeSolvedWY
        \RubikRotation{Y,F}
        \ShowCube{6cm}{0.5}{\DrawRubikCubeLU}
    \end{subfigure}
    \begin{subfigure}
        \centering\RubikCubeSolvedWY
        \RubikRotation{Y,Fp}
        \ShowCube{6cm}{0.5}{\DrawRubikCubeLU}
    \end{subfigure}
    \caption{Exemples de Movimients F y F'}
\end{figure}

\begin{figure}[htbp]
    \centering
    \begin{subfigure}
        \centering\RubikCubeSolvedWY
        \RubikRotation{Y,B}
        \ShowCube{6cm}{0.5}{\DrawRubikCubeLU}
    \end{subfigure}
    \begin{subfigure}
        \centering\RubikCubeSolvedWY
        \RubikRotation{Y,Bp}
        \ShowCube{6cm}{0.5}{\DrawRubikCubeLU}
    \end{subfigure}
    \caption{Exemples de Movimients B y B'}
\end{figure}

\begin{figure}[htbp]
    \centering
    \begin{subfigure}
        \centering\RubikCubeSolvedWY
        \RubikRotation{Y,R}
        \ShowCube{6cm}{0.5}{\DrawRubikCubeLU}
    \end{subfigure}
    \begin{subfigure}
        \centering\RubikCubeSolvedWY
        \RubikRotation{Y,Rp}
        \ShowCube{6cm}{0.5}{\DrawRubikCubeLU}
    \end{subfigure}
    \caption{Exemples de Movimients R y R'}
\end{figure}

\begin{figure}[htbp]
    \centering
    \begin{subfigure}
        \centering\RubikCubeSolvedWY
        \RubikRotation{Y,L}
        \ShowCube{6cm}{0.5}{\DrawRubikCubeLU}
    \end{subfigure}
    \begin{subfigure}
        \centering\RubikCubeSolvedWY
        \RubikRotation{Y,Lp}
        \ShowCube{6cm}{0.5}{\DrawRubikCubeLU}
    \end{subfigure}
    \caption{Exemples de Movimients L y L'}
\end{figure}

\begin{figure}[htbp]
    \centering
    \begin{subfigure}
        \centering\RubikCubeSolvedWY
        \RubikRotation{Y,U}
        \ShowCube{6cm}{0.5}{\DrawRubikCubeLU}
    \end{subfigure}
    \begin{subfigure}
        \centering\RubikCubeSolvedWY
        \RubikRotation{Y,Up}
        \ShowCube{6cm}{0.5}{\DrawRubikCubeLU}
    \end{subfigure}
    \caption{Exemples de Movimients U y U'}
\end{figure}

\begin{figure}[ht!]
    \centering
    \begin{subfigure}
        \centering\RubikCubeSolvedWY
        \RubikRotation{Y,D}
        \ShowCube{6cm}{0.5}{\DrawRubikCubeLU}
    \end{subfigure}
    \begin{subfigure}
        \centering\RubikCubeSolvedWY
        \RubikRotation{Y,Dp}
        \ShowCube{6cm}{0.5}{\DrawRubikCubeLU}
    \end{subfigure}
    \caption{Exemples de Movimients D y D'}
    \label{fig:d-d'}
\end{figure}
