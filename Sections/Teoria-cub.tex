\subsection{Notació dels Moviments}

El cub de rubik es resol gràcies a identificar patrons i executar algoritmes que resolen aquests patrons, aquests algoritmes han d'estar escrits en alguna part per poder-los memoritzar i per això estaà la notació del cub de rubik.
\\\\La notació consta de 6 moviments (F,B,R,L,U,D), que correspon a (Front, Back, Right, Left, Up, Down) que son les respectives direccions en anglés. Per exemple si faig el moviment F gira la capa front la que eta més propera a mi en sentit horari, en canvi si fós F' seria antihorari. En les figures següents es mostra una respresentació gràfica per a cada capa.
\\\\És un concepte difícil d'entendre però de manera simplificada és girar la cara en sentit horari i antihorari desde la cara que vulguis. En les figures següents es mostra una respresentació gràfica per a cada capa.

\begin{figure}[htbp]
    \centering
    \begin{subfigure}
        \centering\RubikCubeSolvedWY
        \RubikRotation{Y,F}
        \ShowCube{7cm}{0.5}{\DrawRubikCubeLU}
    \end{subfigure}
    \begin{subfigure}
        \centering\RubikCubeSolvedWY
        \RubikRotation{Y,Fp}
        \ShowCube{7cm}{0.5}{\DrawRubikCubeLU}
    \end{subfigure}
    \caption{Exemples de Movimients F y F'}
\end{figure}

\begin{figure}[htbp]
    \centering
    \begin{subfigure}
        \centering\RubikCubeSolvedWY
        \RubikRotation{Y,B}
        \ShowCube{7cm}{0.5}{\DrawRubikCubeLU}
    \end{subfigure}
    \begin{subfigure}
        \centering\RubikCubeSolvedWY
        \RubikRotation{Y,Bp}
        \ShowCube{7cm}{0.5}{\DrawRubikCubeLU}
    \end{subfigure}
    \caption{Exemples de Movimients B y B'}
\end{figure}

\begin{figure}[htbp]
    \centering
    \begin{subfigure}
        \centering\RubikCubeSolvedWY
        \RubikRotation{Y,R}
        \ShowCube{7cm}{0.5}{\DrawRubikCubeLU}
    \end{subfigure}
    \begin{subfigure}
        \centering\RubikCubeSolvedWY
        \RubikRotation{Y,Rp}
        \ShowCube{7cm}{0.5}{\DrawRubikCubeLU}
    \end{subfigure}
    \caption{Exemples de Movimients R y R'}
\end{figure}

\begin{figure}[htbp]
    \centering
    \begin{subfigure}
        \centering\RubikCubeSolvedWY
        \RubikRotation{Y,L}
        \ShowCube{7cm}{0.5}{\DrawRubikCubeLU}
    \end{subfigure}
    \begin{subfigure}
        \centering\RubikCubeSolvedWY
        \RubikRotation{Y,Lp}
        \ShowCube{7cm}{0.5}{\DrawRubikCubeLU}
    \end{subfigure}
    \caption{Exemples de Movimients L y L'}
\end{figure}

\begin{figure}[htbp]
    \centering
    \begin{subfigure}
        \centering\RubikCubeSolvedWY
        \RubikRotation{Y,U}
        \ShowCube{7cm}{0.5}{\DrawRubikCubeLU}
    \end{subfigure}
    \begin{subfigure}
        \centering\RubikCubeSolvedWY
        \RubikRotation{Y,Up}
        \ShowCube{7cm}{0.5}{\DrawRubikCubeLU}
    \end{subfigure}
    \caption{Exemples de Movimients U y U'}
\end{figure}

\begin{figure}[htbp]
    \centering
    \begin{subfigure}
        \centering\RubikCubeSolvedWY
        \RubikRotation{Y,D}
        \ShowCube{7cm}{0.5}{\DrawRubikCubeLU}
    \end{subfigure}
    \begin{subfigure}
        \centering\RubikCubeSolvedWY
        \RubikRotation{Y,Dp}
        \ShowCube{7cm}{0.5}{\DrawRubikCubeLU}
    \end{subfigure}
    \caption{Exemples de Movimients D y D'}
\end{figure}
