\pagestyle{normal}
\part{Marc Teòric}
\chapter{Història del Cub de Rubik}

\section{Invent i Introducció (1974-1980)}

El Cub de Rubik, va ser creat per Ernő Rubik el 1974 com una eina d'ensenyament dels
conceptes espacials a estudiants d'arquitectura, es va llançar el 1975 a Budapest amb el nom
"Cub Màgic". El seu disseny original constava de cares de colors sòlids. Aviat, el cub es va estendre per tot el món, però la seva resolució semblava un enigma que estava
a l'abast de poca gent. \cite{Insider} \cite{ross_rubiks}

\section{Primers Intents de Resolució (1980-1981):}

David Singmaster, un estudiant d'enginyeria mecànica a Londres, va desenvolupar la primera notació per descriure els moviments del Cub i va crear el "mètode Singmaster" per resoldre'l,
un fet molt important pel cub. \cite{DavidSingmaster}

\section{Aparició dels Primers Campions (1982-1992):}

A la dècada de 1980 van tenir lloc les primeres competicions de Cub de Rubik oficials , gràcies a
la popularitat que estava agafant. Amb competicions de velocitat que van començar el 1982.
Minh Thai es va convertir en el primer Campió Mundial. Més tard els mètodes de resolució van
evolucionar, i el mètode Friedrich de Jèssica Friedrich es va convertir en un dels més populars.

\section{L'Època dels Speedcubers (2003-2010):}

La World Cube Association (WCA) es va fundar el 2003, establint estàndards i competicions oficials. Es va professionalitzar i van sorgir Speedcubers\footnote{Persones que poden fer el cub de Rubik en molt poc temps} que van elevar els nivells de les competicions.

\section{L'Arribada de 3BLD (2004):}

Més tard es va introduir la categoria de resolució a cegues (3BLD) a la WCA i Tyson Mao es va convertir en el primer campió de 3BLD el 2004.

\section{Actualment(2023):}

El Cub de Rubik i el 3BLD continuen sent una categoria molt emocionant i important en
la comunitat de l'speedcubing i actualment el rècord de 3BLD es troba en 12.10 per Charlie
Eggins d'Austràlia. Actualment s'estan buscant tècniques encara més avançades pel 3BLD,
com algoritmes gairebé el doble d'eficients, però a la vegada el doble de casos. No sabem la
direcció que agafarà el 3BLD, però de segur que ens espera un gran futur.\cite{worldcubeassociation}




