\section{Marc Teòric}
\subsection{Història del Cub de Rubik}

\subsubsection{Invent i Introducció (1974-1980)}

El Cub de Rubik, concebut per Ernő Rubik el 1974 com una eina d'ensenyament per a conceptes espacials, es va llançar el 1975 a Budapest amb el nom "Cub Màgic". El seu disseny original constava de cares de colors sòlids. Aviat, aquest trencaclosques es va estendre per tot el món, però la seva resolució semblava un enigma sense solució clara.

\subsubsection{Primers Intents de Resolució (1980-1981):}

David Singmaster, un estudiant d'enginyeria mecànica a Londres, va desenvolupar la primera notació per descriure els moviments del Cub i va crear el "mètode Singmaster" per resoldre'l, un fita important en la història del cub.

\subsubsection{Aparició dels Primers Campions (1982-1992):}

La dècada de 1980 va veure l'auge de la popularitat del Cub de Rubik, amb competicions de velocitat que van començar el 1982. Minh Thai es va convertir en el primer Campió Mundial. Els mètodes de resolució van evolucionar, i el mètode Friedrich de Jessica Fridrich es va convertir en un dels més populars.

\subsubsection{L'Època dels Speedcubers (2003-2010):}

La World Cube Association (WCA) es va fundar el 2003, establint estàndards i competicions oficials. Speedcubers com Feliks Zemdegs van dominar l'escena, establint rècords mundials en resolució estàndard i altres categories. La resolució del Cub es va convertir en una competició de temps.

\subsubsection{L'Arribada de 3BLD (2003-2009):}

La disciplina de resolució a cegues amb tres capes (3BLD) es va introduir en competicions, on els competidors havien de memoritzar i resoldre el cub sense veure'l. Tyson Mao es va convertir en el primer campió de 3BLD el 2003. Aquesta tècnica es va basar en memoritzar algoritmes i seqüències de moviments.

\subsubsection{Els Temps Cauen Dràsticament (2011-2020):}

Amb l'expansió de comunitats en línia, la tècnica de 3BLD es va perfeccionar i els temps van millorar. Speedcubers com Max Park van establir rècords mundials impressionants en 3BLD. El 2017, Max Park va aconseguir un rècord de 16,80 segons, demostrant el potencial d'aquesta disciplina.

\subsubsection{Avui en Dia (2023 en Endavant):}

El Cub de Rubik i el 3BLD continuen sent una font de passió i competició arreu del món. Les competències atrauen speedcubers de totes les edats i nivells d'habilitat, i la comunitat continua refinant les seves tècniques.




