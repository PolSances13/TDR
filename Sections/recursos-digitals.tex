\chapter{Elaboració d'un PDF tutorial per a principiants}

Ja amb tot els conceptes entesos i la creació de l'app he fet un tutorial per resoldre el cub de Rubik a cegues per a totes les persones que tinguin un domini míním del cub, és a dir, gent que ja sap fer el cub. 
En comparativa amb el contingut del treball, té algunes semblançes ja que tot està explicat de manera simple i conté algunes coses del treball però també inclou mètodes incials i altres conceptes que no he introduit al treball perquè a l'hora de fer el cub sense mirar són essencials i en canvi a l'hora d'entendre el treball no.
\\\\Si hagués de definir el tutorial amb poques paraules, seria que el contingut està explicat com a mi m'hagués agradat que m'ho haguessin explicat. El resultat del tutorial es pot veure a l'annex 2, encara que no està complet ja que quedaria un annex massa llarg al treball, per veure el tutorial complert he d'anar \href{https://polsances13.github.io/roadto3bld/Tutorial.html}{roadto3bld/tutorial}. Dins de la web hi ha les taules de memorització de les 576 combinacions de dues lletres que es poden fer al 3BLD i és per això que és millor que el tutorial es visualitzi a la web.




\chapter{Elaboració d'una web}

Per deixar-ho tot organitzat he decidit fer un web estàtica\footnote{Una pàgina web estàtica és una pàgina web que es mostra al navegador de l'usuari tal i com està emmagatzemada}. La idea darrere de la web és fer com un tipus de central en la qual et porti als diferents contiguts del treball, en resum està feta perquè serveixi com a tutorial a qualsevol que tingui una mica d'experiència en els cubs i es vulgui inciar en el blind.
Està redactada en HTML5 i CSS3, no és una web molt complexa però ha sigut desenvolupada sense gairebé experiència prèvia, només aprenent amb tutorials i llibres d'html es pot visitar a \href{https://polsances13.github.io/roadto3bld/index.html}{roadto3bld}.
\\\\Dins de la web hi ha la pàgina principal on hi ha un presentació del contingut de la web, després tenim un menú que porta a les pestanyes de Tutorial, App i Contacte. 


