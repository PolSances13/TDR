\part*{Introducció}
\pagestyle{Introducció}
\addcontentsline{toc}{chapter}{Introducció}

Vaig descobrir el cub de Rubik quan era petit i a casa feia competicions amb la meva família
de fer una cara del cub, però mai l'havia arribat a fer sencer, algun Nadal m'havien regalat algun
cub més bo, però tenia por a no saber tornar-lo a fer.
\\\\Als 12 anys a primer de l'ESO, un dia vaig decidir aprendre a fer-lo i vaig buscar a YouTube tutorials per poder resoldre'l i ho vaig aconseguir, aquell dia no vaig parar de desfer i fer, va
ser com una connexió amb el cub que em cridava a tornar-lo a fer i a millorar-me a mi mateix.
No vaig tardar a millorar el meu temps i vaig començar a provar cubs nous fins que va arribar
un punt a on vaig perdre la motivació per fer els cubs, ja que aprendre noves categories no
em cridava l'atenció. Vaig intentar fer el cub de Rubik a cegues, però vaig fallar i ho vaig deixar
apart. Llavors va ser a primer de batxillerat escollint tema del TdR que em vaig decantar per
aquest tema, i el treball no només serà aprendre, sinó que també ajudar a tothom a aprendre.
\\\\Durant aquest treball també entro en l'àmbit de la programació, una habilitat que jo no domino i només en tenia una petita base que vaig fer a quart d'ESO, igual que amb els cubs
també he fallat anteriorment en aprendre-ho correctament. En aquest treball utilitzaré els llenguatges de programació Python, HTML, CSS i el sistema de composició de textos LaTeX.
\\\\Així aquest treball busca com a objectiu personal aprendre a fer el cub de Rubik sense mirar
i millorar en l'àmbit de la programació, i com a objectiu del treball, fer un tutorial complet
amb tota mena de recursos que puguin ajudar a tothom a entendre a fer el cub de Rubik sense
mirar.
\\\\La motivació d'aquest treball ha estat aquests intents fallits a l'hora d'aprendre i el "fracàs"
que havia experimentat.
\\\\Per fer aquest treball necessitaré un ordinador amb connexió a internet, el meu cub 3x3 i
la meva passió pels cubs.




