\pagestyle{conclusions}
\part*{Conclusions}
\addcontentsline{toc}{chapter}{Conclusions}

L'objectiu principal del treball era realitzar un tutorial complet per realitzar el cub de Rubik a cegues per a totohm, i ha estat assolit completament ja que el tutorial que he realitzat conté conceptes que van desde el inicis amb el cub, la notació fins a mètodes intermitjos que s'utlitzen a competicions oficials. 
No només s'ha realitzat un tutorial, sinó que s'ha creat el que podria ser una "plataforma" per aprendre a fer el cub, ja que la web serveix de concentrador del tutorial i la aplicació per ajudar a memoritzar.
\\\\Els objectius personals del treball, també han sigut assolits, ja que he aconseguit aprendre a fer el cub de rubik de sense mirar i també he aprofundit en l'àmbit de la programació. He realitzat una aplicació amb python que té interfície gràfica, cosa que mai se m'havia passat pel cap, amb Html he realitzat una web estàtica, que no és la web més complexa pero jo no comptava amb cap coneixemnt d'HTML. Finalment la redacció del document fet amb LaTeX, és juntament amb el fet de fer el cub a cegues, el més important del treball. 
\\\\Durant el treball, he tingut molts problemes, sobretot amb la programació, ja que a l'hora d'aprendre el procés és prova i error. Amb el cub també he tingut alguns problemes però no tants com amb la programació.
